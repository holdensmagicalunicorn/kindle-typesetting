\documentclass[11pt,letterpaper]{article}
\usepackage[paperwidth=3.6in,paperheight=4.5in,     % 3.6in x 4.8in physical
  total={3.1in,4.0in},
  includefoot]{geometry}

\begin{document}
\sloppypar
And now comes the good news, if you haven't used
computer typesetting before. You don't have to worry about where to
break lines in a paragraph (i.e., where to stop at the right margin
and being a new line), because \TeX\ will do that for you.
Your manuscript file can contain long lines or short lines, or both;
it doesn't matter.
This is especially helpful when you make changes, since you
don't have to retype anything except the words that changed.
\textit{Every time you being a new line in your manuscript file 
it is essentially the same as typing a space.}
When \TeX\ has read an entire paragraph --- in this case lines
7 to 11 --- it will try to break up the text so that each line of output,
except the last, contains about the same amount of copy; and it will
hyphenate words if necessary to keep the spacing consistent, but only
as a last resort.
\newpage

\begin{verbatim}
Once upon a time, in a distant
  galaxy called ---
there lived a computer
\end{verbatim}

Consider also the following examples, which show that binary operations
can be used as ordinary symbols in superscripts and subscripts:

\begin{itemize}
\item $f^*(x) \cap f_*(y)$
\item $g^\circ \mapsto g^\bullet$
\end{itemize}
\end{document}

