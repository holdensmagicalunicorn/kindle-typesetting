\documentclass[11pt,letterpaper]{article}
\usepackage[paperwidth=3.6in,paperheight=4.5in,     % 3.6in x 4.8in physical
  total={3.1in,4.0in},
  includefoot]{geometry}

\begin{document}
\sloppypar
And now comes the good news, if you haven't used
computer typesetting before. You don't have to worry about where to
break lines in a paragraph.
\textit{Every time you being a new line in your manuscript file 
it is essentially the same as typing a space.}

\begin{verbatim}
Once upon a time, in a distant
there lived a computer
\end{verbatim}

Consider also the following examples, which show that binary operations
can be used as ordinary symbols in superscripts and subscripts:

\begin{itemize}
\item $f^*(x) \cap f_*(y)$
\item $g^\circ \mapsto g^\bullet$
\end{itemize}
\end{document}

